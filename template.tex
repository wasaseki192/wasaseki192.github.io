\documentclass[dvipdfmx,autodetect-engine]{jsarticle}

\usepackage[dvipdfmx]{graphicx} % 画像挿入
\usepackage{here} % 画像や表の表示位置の矯正

\usepackage{tikz} % 図
\usepackage{tikz-3dplot} % 3次元の図
\usetikzlibrary{arrows.meta} % 矢印
\usepackage{tcolorbox} % 枠
\tcbuselibrary{breakable, skins, theorems} % 枠

\usepackage{amssymb} % 数式
\usepackage{amsmath} % 数式
\usepackage{braket} % ブラケット記法
\usepackage{physics} % 物理の数式
\usepackage{amsthm} % 定理

\usepackage{bm} % ベクトル(太字)
\usepackage{amsfonts} % フォント
\usepackage{mathrsfs} % 花文字

\usepackage[dvipdfmx,colorlinks=true]{hyperref} % リンクを付ける

\pagestyle{plain}
 
\makeatletter

% 表紙に表示するものを定義
\def\@thesis{論文名}
\def\id#1{\def\@id{#1}}
\def\master#1{\def\@master{#1}}
\def\department#1{\def\@department{#1}}

% 表紙全体を定義
\def\@maketitle{
\begin{center}
{\huge \@thesis \par}
\vspace{10mm}
{\LARGE\bf \@title \par}
\vspace{20mm}
{\Large 実験日 NNNN年NN月NN日\par}
\vspace{10mm}
{\Large 提出日 \today\par}
\vspace{20mm}
{\Large \@department \par}
\vspace{10mm}
{\Large \@master \par}
\vspace{20mm}
{\Large \hspace{14mm}氏名 \@author\par}
\vspace{10mm}
{\Large 共同実験者 学籍番号 ○○ ○○\par}
{\Large \hspace{24mm} 学籍番号 ○○ ○○\par}
%{\Large \hspace{24mm} 学籍番号 ○○ ○○\par}
%{\Large \hspace{24mm} 学籍番号 ○○ ○○\par}
\end{center}
}
% 自動的にコンパイルした日付になる設定なので注意
 
\makeatother

% 表紙に表示する内容 
\title{実験タイトル}
\department{○○大学○○学部○○学科}
\master{○○課程N年}
\author{学籍番号 ○○ ○○}

\begin{document}

% 表紙を表示
\maketitle
\thispagestyle{empty}
\newpage

\section{目的}
目的

\section{実験装置}
\begin{itemize}
    \item 実験装置
\end{itemize}

\section{実験手順}
\begin{enumerate}
    \item 実験手順
\end{enumerate}

\section{結果}
結果

\section{考察}
考察

\section{まとめ}
まとめ

\begin{thebibliography}{99} %参考文献の数以上の値を入力
\bibitem{1} 参考文献
\end{thebibliography}
% \cite{1}とすることで本文中で引用できる



% コピペ用

% 表
% \begin{table}[H]
%   \centering
%   \begin{tabular}{|c|c|}
%   \hline
%   左上 & 左下 \\ \hline
%   左下 & 右下 \\ \hline
%   \end{tabular}
% \end{table}

% 画像
% \begin{figure}[H]
%   \begin{center}
%    \includegraphics[scale=1]{ファイル名}
%   \end{center}
% \end{figure}

\end{document}